% !TEX root = paper/paper.tex
\vspace{-1em}
\begin{abstract}
Humans are capable of perceiving a scene at a glance, and obtain deeper understanding with additional time. Similarly, visual recognition deployments should be robust to varying computational budgets. Such situations require Anytime recognition ability, which is rarely considered in computer vision research. We present a method for learning dynamic policies to optimize Anytime performance in visual architectures. Our model sequentially orders feature computation and performs subsequent classification. Crucially, decisions are made at test time and depend on observed data and intermediate results. We show the applicability of this system to standard problems in scene and object recognition. On suitable datasets, we can incorporate a semantic back-off strategy that gives maximally specific predictions for a desired level of accuracy; this provides a new view on the time course of human visual perception.
\end{abstract}

% !TEX root = paper/paper.tex
\vspace{-1.5em}
\section{Introduction}

Anytime recognition is a core competence in human perception, mediating between reflexive recognition and deep analysis of visual input.
Human studies have produced evidence for coarse-to-fine processing of visual input as more time becomes available \cite{Fei-Fei-Vision-2007,mace09plos}.
The underlying mechanisms are unknown, with only a few attempts to explain the temporal dynamics (e.g. via sequential decision processes \cite{hegde08neuro}).
% kadar12vision was cited alongside fei fei

While multi-class recognition in computer vision has achieved levels of performance that allow useful real-world implementation, state-of-the-art methods tend to be computationally expensive and insensitive to Anytime demands.
As these methods are applied at scale, managing their resource consumption (power or cpu-time) cost becomes increasingly important.
For tasks such as personal robotics, the ability to deploy varying levels of processing to different stimuli, depending on computational demands on the robot, also seems crucial.

For most state-of-the-art classification methods, different features are extracted from an image instance at different costs, and contribute differently to decreasing classification error.
Although ``the more features, the better'', high accuracy can be achieved with only a small subset of features for some instances---and different instances benefit from different subsets of features.
For example, simple binary features are sufficient to quickly detect faces \cite{Viola2004} but not more varied visual objects, while the features most useful for separating landscapes from indoor scenes \cite{Xiao-CVPR-2010} are different from those most useful for recognizing fine distinctions between bird species \cite{Farrell-ICCV-2011}.

Computing all features for all images is infeasible in a deployment sensitive to Anytime needs, as each feature brings a significant computational burden.
To deal with this problem, we can set an explicit cost \emph{budget}, specified in terms of wall time or total power expended or another metric.
Additionally, we strive for \emph{Anytime} performance---the ability to terminate the classifier even before the cost budget is depleted and still obtain the best answer.
In this paper, we address the problem of selecting and combining a subset of features under an \emph{Anytime} cost budget.

To exploit the fact that different instances benefit from different subsets of features, our approach to feature selection is a sequential policy.
To learn the policy parameters, we formulate the problem as a Markov Decision Process (MDP) and use reinforcement learning methods.
With different settings of parameters, we can learn policies ranging from \textbf{Static, Myopic}---greedy selection not relying on any observed feature values, to \textbf{Dynamic, Non-myopic}---relying on observed values and considering future actions.

Since test-time efficiency is our motivation, our methods should carry little computational burden.
For this reason, our models are based on linear evaluations, not nearest-neighbor or graphical model methods.
Because different features can be selected for different instances, and because our system may be called upon to give an answer at any point during its execution, the feature combination method needs to be robust to a large number of different observed-feature subsets.
To this end, we present a novel method for learning several classifiers for different clusters of observed-feature subsets.

We evaluate our method on multi-class recognition tasks.
We first demonstrate on synthetic data that our algorithm learns to pick features most useful for the specific test instance.
We demonstrate the advantage of non-myopic over greedy, and of dynamic over static on this and the Scene-15 visual classification dataset.
Then we show results on a subset of the hierarchical ImageNet dataset, where we additionally learn to provide the most specific answers for any desired cost budget and accuracy level.

\section{Recognition Problems and Related Work}

Formally, we deal with a dataset of images $\mathcal{D}$, where each image $\mathcal{I}$ contains zero or more objects.
Each object is labeled with exactly one category label $k \in \{1, \dots, K\}$.

The multi-class, multi-label \textbf{classification} problem asks whether $\mathcal{I}$ contains at least one object of class $k$.
We write the ground truth for an image as $\mathbf{C}=\{C_1,\dots,C_K\}$, where $C_k \in \{0,1\}$ is set to $1$ if an object of class $k$ is present.

The \textbf{detection} problem is to output a list of bounding boxes (sub-images defined by four coordinates), each with a real-valued confidence that it encloses a single instance of an object of class $k$, for each $k$.
The answer for a single class $k$ is given by an algorithm $\emph{detect}(\mathcal{I},k)$, which outputs a list of sub-image bounding boxes $B$ and their associated confidences.

Performance is evaluated by plotting precision vs. recall across dataset $\mathcal{D}$ (by progressively lowering the confidence threshold for a positive detection).
The area under the curve yields the Average Precision (AP) metric, which has become the standard evaluation for recognition performance on challenging datasets in vision \cite{pascal-voc-2010}.
A common measure of a correct detection is the PASCAL overlap: two bounding boxes are considered to match if they have the same class label and the ratio of their intersection to their union is at least $\frac{1}{2}$.

To highlight the hierarchical structure of these problems, we note that the confidences for each sub-image $b \in B$ may be given by $\emph{classify}(b,k)$, and, more saliently for our setup, correct answer to the detection problem also answers the classification problem.

Multi-class performance is evaluated by averaging the individual per-class AP values.
In a specialized system such as the advertising case study from~\autoref{sec:introduction}, the metric generalizes to a weighted average, with the weights set by the \emph{values} of the classes.

\subsection{Related Work}

\myparagraph{Object detection}
The best recent performance has come from detectors that use gradient-based features to represent objects as either a collection of local patches or as object-sized windows \cite{Dalal2005,Lowe2004}.
Classifiers are then used to distinguish between featurizations of a given class and all other possible contents of an image window.
Window proposal is most often done exhaustively over the image space, as a ``sliding window''.

For state-of-the-art performance, the object-sized window models are augmented with parts \cite{Felzenszwalb2010a}, and the bag-of-visual-words models employ non-linear classifiers \cite{Vedaldi2009}.
We employ the widely used Deformable Part Model detector \cite{Felzenszwalb2010a} in our evaluation.
%Some approaches use ``jump windows'' (hypotheses voted on by local features) \cite{Vedaldi2009,Vijayanarasimhan2011}, or a bounded search over the space of all possible windows \cite{Lampert2008a}.

%None of the best-performing systems treat window proposal and evaluation as a closed-loop system, with feedback from evaluation to proposal.
%Some work has been done on this topic, mostly inspired by ideas from biological vision and attention research~\cite{Butko2009,Vogel2008}.

% Most detection methods train individual models for each class.
% Work on inherently multi-class detection focuses largely on making detection time sublinear in the number of classes through sharing features \cite{Torralba2007,Fan2005}.

% A post-processing extension to detection systems uses structured prediction to incorporate multi-class context as a principled replacement for non-maximum suppression \cite{Desai2009}.

\myparagraph{Using context}
The most common source of context for detection is the \emph{scene} or other non-detector cues; the most common scene-level feature is the GIST \cite{Oliva2001a} of the image.
We use this source of scene context in our evaluation.

Inter-object context has also been shown to improve detection \cite{Torralba2004}.
In a standard evaluation setup, inter-object context plays a role only in post-filtering, once all detectors have been run.
In contrast, our work leverages inter-object context in the action-planning loop.

A critical summary of the main approaches to using context for object and scene recognition is given in \cite{Galleguillos2010}.
For the commonly used PASCAL VOC dataset \cite{pascal-voc-2010}, GIST and other sources of context are quantitatively explored in~\cite{Divvala2009}.

\myparagraph{Efficiency through cascades}
An early success in efficient object detection of a single class uses simple, fast features to build up a \emph{cascade} of classifiers, which then considers image regions in a sliding window regime \cite{Viola2001}.
Most recently, cyclic optimization has been applied to optimize cascades with respect to feature computation cost as well as classifier performance \cite{Chen2012}.

Cascades are not dynamic policies: they cannot change the order of execution based on observations obtained during execution, which is our goal.

\myparagraph{Anytime and active classification}
This surprisingly little-explored line of work in vision is closest to our approach.
A recent application to the problem of visual detection picks features with maximum value of information in a Hough-voting framework \cite{Vijayanarasimhan2010}. 
There has also been work on active classification \cite{Gao2011} and active sensing \cite{Yu2009}, in which intermediate results are considered in order to decide on the next classification step.
Most commonly, the scheduling in these approaches is greedy with respect to some manual quantity such as expected information gain.
In contrast, we learn policies that take actions without any immediate reward.

\input{../method_new}
\section{Time-sensitive evaluation} \label{sec:evaluation}
Average Precision (AP) has become a standard evaluation for detector performance on challenging datasets.
AP is the area under the Precision vs. Recall curve, which is obtained by varying the threshold on the confidence of the detector.

Just as we care about the performance of our system at different thresholds of detection, we should also care about performance as the system is given different amount of time to output detections.
Accordingly, we plot the P-R \emph{surface} instead of the P-R curve, with time as one of the axes.
The surface can be integrated along the Recall dimension to yield an AP vs. time curve.

The goal of our scheme is to schedule computation optimally, so that the largest part of the detection performance is recovered early on.
In the limit of infinite time, we still want optimal performance---the performance of our system should not degrade as it is given more time.
Accordingly, our goal is \emph{Anytime} performance starting at some fixed deadline.

As our task is fundamentally in \emph{multi-class} object detection, we rely on a slightly different evaluation than is commonly used: instead of pooling detections across images in the dataset but not classes, we pool detections across classes, but evaluate per-image (and report the average).
This has been done before~\cite{Desai2009}.
%!TEX root=paper/thesis.tex
\chapter{Conclusion}

\section{NIPS12 Conclusion}
We presented a method for learning ``closed-loop'' policies for multi-class object recognition, given existing object detectors and classifiers and a metric to optimize.
The method learns the optimal policy using reinforcement learning, by observing execution traces in training.
If detection on an image is cut off after only half the detectors have been run, our method does $66\%$ better than a random ordering, and $14\%$ better than an intelligent baseline.
In particular, our method learns to take action with no intermediate reward in order to improve the overall performance of the system.

As always with reinforcement learning problems, defining the reward function requires some manual work.
Here, we derive it for the novel detection AP vs. Time evaluation that we suggest is useful for evaluating efficiency in recognition.
Although computation devoted to scheduling actions is less significant than the computation due to running the actions, the next research direction is to explicitly consider this decision-making cost; the same goes for feature computation costs.
Additionally, it is interesting to consider actions defined not just by object category but also by spatial region.
The code for our method is available\footnote{\url{http://sergeykarayev.com/work/timely/}}.

\section{CVPR14 Conclusion}
We have shown how to optimize feature selection and classification strategies under an Anytime objective by modeling the associated process as a Markov Decision Process.
Throughout the experiments we show how strategies that adapt the course of computation at test time lead to gains in performance and efficiency.
Beyond the aspects of practical deployment of vision systems that our work is motivated by, we are curious to further investigate our model as a tool to study human cognition and the time course of visual perception.

Lastly, the recent successes of convolutional neural nets for visual recognition open an exciting new avenue for exploring cost-sensitivity.
Layers of a deep network can be seen as features in our system, through which a properly learned policy can optimally direct computation.


{\small
\section*{Acknowledgements}
This research was supported by the National Defense Science and Engineering Graduate Fellowship; DARPA Mind's Eye and MSEE programs; NSF awards IIS-0905647, IIS-1134072, and IIS-1212798; by Toyota; and by the Intel Visual Computing Institute.

\bibliographystyle{ieee}
\bibliography{../../sergeyk_library}
}
