\documentclass{article} % For LaTeX2e
\usepackage{nips12submit_e,times}
\usepackage{graphicx}
\usepackage{amsmath,amssymb} % define this before the line numbering.
\usepackage{color}
\usepackage{subfig}
\usepackage{natbib}
\usepackage{array}
\usepackage{tabularx}
\usepackage[pagebackref=true,breaklinks=true,colorlinks,bookmarks=false]{hyperref}

\newcommand{\argmax}{\operatornamewithlimits{argmax}}
\def\subsectionautorefname{section}
\definecolor{light-gray}{gray}{0.5}
\newcommand{\aside}[1]{\textcolor{light-gray}{\emph{#1}}}
\newcommand{\todo}[1]{\textcolor{red}{\emph{#1}}}
\newcommand{\cut}[1]{\textcolor{light-gray}{#1}}
\newcommand{\comment}[1]{}

\newcommand{\myparagraph}[1]{\vspace{-0.1cm}{\bf #1}}

\title{Timely Object Recognition}

\author{
Sergey Karayev \\
UC Berkeley
\And
Tobias Baumgartner \\
RWTH Aachen University
\And
Mario Fritz \\
MPI for Informatics 
\And
Trevor Darrell \\
UC Berkeley
}

\newcommand{\fix}{\marginpar{FIX}}
\newcommand{\new}{\marginpar{NEW}}

\nipsfinalcopy % Uncomment for camera-ready version

\begin{document}
\maketitle

\begin{abstract}
In a large visual multi-class detection framework, the timeliness of results can be crucial.
Our method for \emph{timely} multi-class detection aims to give the best possible performance at any single point after a start time; it is terminated at a deadline time.
Toward this goal, we formulate a dynamic, closed-loop policy that infers the contents of the image in order to decide which detector to deploy next.
In contrast to previous work, our method significantly diverges from the predominant greedy strategies, and is able to learn to take actions with deferred values.
We evaluate our method with a novel \emph{timeliness} measure, computed as the area under an Average Precision vs. Time curve.
Experiments are conducted on the PASCAL VOC object detection dataset.
If execution is stopped when only half the detectors have been run, our method obtains $66\%$ better AP than a random ordering, and $14\%$ better performance than an intelligent baseline.
On the timeliness measure, our method obtains at least $11\%$ better performance.
Our method is easily extensible, as it treats detectors and classifiers as black boxes and learns from execution traces using reinforcement learning.
\end{abstract}

%!TEX root=paper/paper.tex
\chapter{Introduction}\label{sec:introduction}

\section{Motivation}

\PM{Perception}
It is well known that human perception is both Anytime, meaning that a scene can be described after even a short presentation, and progressive, meaning that the quality of description increases with more time.
The progressive time course of visual perception has been confirmed by multiple studies \parencite{Vanrullen-1996,Fei-Fei-Vision-2007}, with some studies providing evidence that enhancement occurs in an ontologically meaningful way.
For example, people tend to recognize something as an animal before recognizing it as a dog \parencite{Mace-PloS-2009}.
The underlying mechanisms of this behavior are not well explored, with only a few attempts made to explain the temporal dynamics --- for instance, a promising work by \cite{Hegde-Neuro-2008} has employed the framework of sequential decision processes.

\PM{Computer applications}
Meanwhile, automated visual recognition has achieved levels of performance that allow useful real-world implementation.
We focus on two problem formulations: \emph{image classification}, in which some property of the image -- such as scene type, visual style, or even object presence -- is predicted, and \emph{object detection}, in which the location and category (or identity) of all objects in a scene is predicted.
Solutions to the two problems are often linked, as classification can be a ``subroutine'' in a detection method.
State-of-the-art methods for classification and detection tend to be computationally expensive, insensitive to Anytime demands, and not progressively enhanced.

\PM{Application}
As real-world deployment of recognition methods grows, managing resource cost (power or compute time) becomes increasingly important.
For tasks such as personal robotics, it is crucial to be able to deploy varying levels of processing to different stimuli, depending on computational demands on the robot.
A hypothetical system for vision-based advertising, in which paying customers engage with the system to have their products detected in images on the internet, presents another example.
The system has different values (in terms of cost per click) and accuracies for different classes of objects, and the backlog of unprocessed images fluctuates based on demand and available server time.
A recognition strategy to maximize profit in such an environment should exploit all signals available to it, and the quality of detections should be Anytime, depending on the length of the queue (for example, lowering recall with increased queue pressure).

\PM{Visual Features \& Classification}
For most state-of-the-art classification methods, a range of features are extracted from an image instance and used to train a classifier.
Since the feature vectors are usually high-dimensional, linear classification methods are used most often.
Features are extracted at different costs, and contribute differently to decreasing classification error.
Although it can generally be said that ``the more features, the better,'' high accuracy can of course be achieved with only a small subset of features for some instances.
Additionally, different instances benefit from different subsets of features.
For example, simple binary features are sufficient to quickly detect faces \parencite{Viola-IJCV-2004} but not more varied visual objects, while the features most useful for separating landscapes from indoor scenes \parencite{Xiao-CVPR-2010} are different from those most useful for recognizing fine distinctions between bird species \parencite{Farrell-ICCV-2011}.
\autoref{fig:features} presents several common visual features.

\PM{Detection}
Detection methods tend to employ the same visual features and classifiers but apply them to many image sub-regions.
Approaches can broadly be grouped into (A) \emph{per-class, all-region}, (B) \emph{all-class, all-region}, and (C) \emph{all-class, proposed-region} methods.
State-of-the-art \emph{all-class, proposed-region} methods such as \cite{Girshick-CVPR-2014} and \emph{per-class, all-region} methods such as \cite{Felzenszwalb2010a} are considerably slow, performing an expensive computation on (respectively) a thousand to a million image windows.
To maximize early performance gains of these methods, scene and inter-object contextual cues can be exploited in two ways.
First, regions can be processed in an intelligent order, with most likely locations selected first.
Second, if detectors are applied per class, then they can be sequenced so as to maximize the chance of finding objects actually present in the image.
And even the most recent \emph{all-class, all-region}, Convolutional Neural Net (CNN)-based detection methods such as \cite{He-ECCV-2014}, which take advantage of high-performance convolutional primitives for region processing and detect for all classes simultaneously, can be sped up using our idea of cascaded classification.

\section{Our Contributions}

\PM{Costliness}
Computing all features, running all detectors, or processing all regions for all images is infeasible in a deployment sensitive to Anytime needs, as each feature brings a significant computational burden.
Yet the conventional approach to evaluating visual recognition does not consider efficiency, and evaluates performance independently across classes.
We address the problem of selecting and combining a subset of features under an Anytime cost budget (specified in terms of wall time or total power expended or another metric) and propose a new \emph{costliness} measure of performance vs. cost.

\PM{Learning a Policy}
To exploit the fact that different instances benefit from different subsets of features, our approach to feature selection is a sequential policy.
To learn the policy parameters, we formulate the problem as a Markov Decision Process (MDP) and use reinforcement learning methods.
The method does not make many assumptions about the underlying actions, which can be existing object detectors and feature-specific classifiers.
With different settings of parameters, we can learn policies ranging from \textbf{Static, Myopic}---greedy selection not relying on any observed feature values, to \textbf{Dynamic, Non-myopic}---relying on observed values and considering future actions.
The foundational machinery is laid out in \autoref{sec:det_method}.

\PM{Per-class Detection}
For \emph{per-class} detection, the actions are time-consuming detectors applied to the whole image, as well as a quick scene classifier.
We run scene context and object class detectors over the whole image sequentially, using the results of detection obtained so far to select the next actions.
Since the actions are time-consuming, we use a powerful inference mechanism to select the best next action.
In \autoref{sec:det_evaluation}, we evaluate on the PASCAL VOC dataset and obtain better performance than all baselines when there is less time available than is needed to exhaustively run all detectors.
This work was originally presented in \cite{Karayev-NIPS-2012} and all work is open source\footnote{Available at \url{https://github.com/sergeyk/timely_object_recognition}}.

\PM{Image Classification}
Classification actions are much faster than detectors, and the action-selection method accordingly needs to be fast.
Because different features can be selected for different instances, and because our system may be called upon to give an answer at any point during its execution, the feature combination method needs to be robust to a large number of different observed-feature subsets.
In \autoref{sec:clf_chapter}, we consider several value-imputation methods and present a method for learning several classifiers for different clusters of observed-feature subsets.
We first demonstrate on synthetic data that our algorithm learns to pick features most useful for the specific test instance.
We demonstrate the advantage of non-myopic over greedy, and of dynamic over static on this and the Scene-15 visual classification dataset.
Then we show results on a subset of the hierarchical ImageNet dataset, where we additionally learn to provide the most specific answers for any desired cost budget and accuracy level.
This work was originally presented in \cite{Karayev-CVPR-2014} and all work is open source\footnote{Available at \url{https://github.com/sergeyk/anytime_recognition}}.

\PM{Cascade CNN}
We additionally investigate a novel approach for speeding up a state-of-the-art CNN-based detection method, and propose a general technique for accelerating CNNs applied to class imbalanced data.
We employ the classic idea of the cascade by inserting a \emph{reject} option between expensive convolutional layers.
When a CNN processes batches of images, which is standard for many applications, the reject layers allows ``thinning'' of the batch as it progresses through the network, thus saving processing time.
This method is applicable to both \emph{all-class, proposed-region} methods such as \cite{Girshick-CVPR-2014} and \emph{all-class, all-region} methods such as \cite{He-ECCV-2014}.
We demonstrate results --- along with a variety of strong baselines -- on the former method, and show that the Cascade CNN method obtains a nearly 10x speed-up with only marginal drop in accuracy.
All work is reported in \autoref{sec:ccnn_chapter}.

\PM{Recognizing Style}
Lastly, in \autoref{sec:style_chapter} we present two novel datasets and first results for an underexplored research problem in computer vision -- recognizing visual style.
In preparation for an Anytime approach, we evaluate several different features (including CNNs) for the task, and explore content-style correlations in our datasets.
Our large-scale learning gives state-of-the-art results on an existing dataset of image quality and photographic style, and provides a strong baseline on our contributed datasets of 80K photos and 85K paintings labeled with their style and genre.
In a demonstration of cross-dataset understanding of style, we show how results of a search by content can be filtered by style.
This work was originally presented in \cite{Karayev-BMVC-2014}, and all code is open source\footnote{Available at \url{https://github.com/sergeyk/vislab}}.

\PM{Future Directions}
This thesis provides an effective foundation for Anytime visual recognition, and points the way to interesting further developments.
Our MDP-based formulation of learning a feature-selection policy is empirically effective, but heuristic in nature.
The recently developed framework of adaptive submodularity \parencite{Golovin-and-Krause-2010-JAIR} could provide theoretical near-optimality results for some policies, but developing an appropriate objective for our task is not straightforward.
We showed our Cascade CNN model to be effective for a region-based detection task -- but the model was not trained end-to-end with the threshold layers.
An even more interesting future development would add an Anytime loss layer that combines classification output from multiple levels of the network in a cost-sensitive way.
We expand on these ideas in \autoref{sec:conclusion}.

\section{Recognition Problems and Related Work}

Formally, we deal with a dataset of images $\mathcal{D}$, where each image $\mathcal{I}$ contains zero or more objects.
Each object is labeled with exactly one category label $k \in \{1, \dots, K\}$.

The multi-class, multi-label \textbf{classification} problem asks whether $\mathcal{I}$ contains at least one object of class $k$.
We write the ground truth for an image as $\mathbf{C}=\{C_1,\dots,C_K\}$, where $C_k \in \{0,1\}$ is set to $1$ if an object of class $k$ is present.

The \textbf{detection} problem is to output a list of bounding boxes (sub-images defined by four coordinates), each with a real-valued confidence that it encloses a single instance of an object of class $k$, for each $k$.
The answer for a single class $k$ is given by an algorithm $\emph{detect}(\mathcal{I},k)$, which outputs a list of sub-image bounding boxes $B$ and their associated confidences.

Performance is evaluated by plotting precision vs. recall across dataset $\mathcal{D}$ (by progressively lowering the confidence threshold for a positive detection).
The area under the curve yields the Average Precision (AP) metric, which has become the standard evaluation for recognition performance on challenging datasets in vision \cite{pascal-voc-2010}.
A common measure of a correct detection is the PASCAL overlap: two bounding boxes are considered to match if they have the same class label and the ratio of their intersection to their union is at least $\frac{1}{2}$.

To highlight the hierarchical structure of these problems, we note that the confidences for each sub-image $b \in B$ may be given by $\emph{classify}(b,k)$, and, more saliently for our setup, correct answer to the detection problem also answers the classification problem.

Multi-class performance is evaluated by averaging the individual per-class AP values.
In a specialized system such as the advertising case study from~\autoref{sec:introduction}, the metric generalizes to a weighted average, with the weights set by the \emph{values} of the classes.

\subsection{Related Work}

\myparagraph{Object detection}
The best recent performance has come from detectors that use gradient-based features to represent objects as either a collection of local patches or as object-sized windows \cite{Dalal2005,Lowe2004}.
Classifiers are then used to distinguish between featurizations of a given class and all other possible contents of an image window.
Window proposal is most often done exhaustively over the image space, as a ``sliding window''.

For state-of-the-art performance, the object-sized window models are augmented with parts \cite{Felzenszwalb2010a}, and the bag-of-visual-words models employ non-linear classifiers \cite{Vedaldi2009}.
We employ the widely used Deformable Part Model detector \cite{Felzenszwalb2010a} in our evaluation.
%Some approaches use ``jump windows'' (hypotheses voted on by local features) \cite{Vedaldi2009,Vijayanarasimhan2011}, or a bounded search over the space of all possible windows \cite{Lampert2008a}.

%None of the best-performing systems treat window proposal and evaluation as a closed-loop system, with feedback from evaluation to proposal.
%Some work has been done on this topic, mostly inspired by ideas from biological vision and attention research~\cite{Butko2009,Vogel2008}.

% Most detection methods train individual models for each class.
% Work on inherently multi-class detection focuses largely on making detection time sublinear in the number of classes through sharing features \cite{Torralba2007,Fan2005}.

% A post-processing extension to detection systems uses structured prediction to incorporate multi-class context as a principled replacement for non-maximum suppression \cite{Desai2009}.

\myparagraph{Using context}
The most common source of context for detection is the \emph{scene} or other non-detector cues; the most common scene-level feature is the GIST \cite{Oliva2001a} of the image.
We use this source of scene context in our evaluation.

Inter-object context has also been shown to improve detection \cite{Torralba2004}.
In a standard evaluation setup, inter-object context plays a role only in post-filtering, once all detectors have been run.
In contrast, our work leverages inter-object context in the action-planning loop.

A critical summary of the main approaches to using context for object and scene recognition is given in \cite{Galleguillos2010}.
For the commonly used PASCAL VOC dataset \cite{pascal-voc-2010}, GIST and other sources of context are quantitatively explored in~\cite{Divvala2009}.

\myparagraph{Efficiency through cascades}
An early success in efficient object detection of a single class uses simple, fast features to build up a \emph{cascade} of classifiers, which then considers image regions in a sliding window regime \cite{Viola2001}.
Most recently, cyclic optimization has been applied to optimize cascades with respect to feature computation cost as well as classifier performance \cite{Chen2012}.

Cascades are not dynamic policies: they cannot change the order of execution based on observations obtained during execution, which is our goal.

\myparagraph{Anytime and active classification}
This surprisingly little-explored line of work in vision is closest to our approach.
A recent application to the problem of visual detection picks features with maximum value of information in a Hough-voting framework \cite{Vijayanarasimhan2010}. 
There has also been work on active classification \cite{Gao2011} and active sensing \cite{Yu2009}, in which intermediate results are considered in order to decide on the next classification step.
Most commonly, the scheduling in these approaches is greedy with respect to some manual quantity such as expected information gain.
In contrast, we learn policies that take actions without any immediate reward.

\section{Recognition Problems}

We deal with a dataset of images $\mathcal{D}$, where each image $\mathcal{I}$ contains at least one, and often multiple, objects.
Each object is labeled with exactly one category label $k \in \{1, \dots, K\}$.

The multi-class, multi-label \textbf{classification} problem asks whether $\mathcal{I}$ contains at least one object of class $k$.
The answer for a single label is given with a real-valued confidence by a function $\emph{classify}(\mathcal{I},k)$.
We write the ground truth for an image as $\mathbf{C}=\{C_1,\dots,C_K\}$, where $C_k \in \mathbb{B} = \{0,1\}$ is set to $1$ if an object of class $k$ is present.

The answer is evaluated by plotting precision vs. recall across dataset $\mathcal{D}$ (by progressively lowering the confidence threshold for a positive label) and integrating to yield the Average Precision (AP) metric, which has become the standard evaluation for recognition performance on challenging datasets \cite{pascal-voc-2010}.

\comment{
We can make the classification problem more difficult by posing the \emph{counting} problem, which asks how many objects of class $k$ are present in $\mathcal{I}$, for each $k$.
This setting is not commonly evaluated; we mention it for its usefulness in later exposition.
}

The \textbf{detection} problem is to output a list of bounding boxes (sub-images defined by four coordinates), each with a real-valued confidence that it encloses a single instance of an object of class $k$, for each $k$.
The answer for a single class is given by an algorithm $\emph{detect}(\mathcal{I},k)$, which outputs a list of sub-image bounding boxes $B$ and their associated confidences.

A common measure of a correct detection is the PASCAL overlap: two bounding boxes are considered to match if they have the same label and the ratio of their intersection to their union is at least $\frac{1}{2}$.
Again, Average Precision is the single-number metric for the performance of a detector.

As our task is fundamentally in \emph{multi-class} object detection, we rely on a slightly different evaluation than is commonly used (although it has precedent in \cite{Desai2009}).
Instead of pooling detections across images in the dataset, and considering classes individually, we pool detections across classes, but consider images individually, reporting results averaged across the dataset.

To highlight the hierarchical structure of these problems, we note that (1) the confidences for each sub-image $b \in B$ may be given by $\emph{classify}(b,k)$; (2) the correct answer to the detection problem also answers the classification problem. \comment{the counting and classification problems}

Our goal is a general recognition policy that outputs both classification and detection results; we evaluate on both tasks.

\section{Multi-class Recognition Policy} \label{sec:tech}
\begin{figure}[h!]
\center{\includegraphics[width=0.66\linewidth]
    {../figures/pomdp.png}}
  \caption{Summary of our approach to the problem. Our system has two major parts: (1) selecting an action by predicting its value; (2) updating the belief state with observations resulting from the action.}
  \label{fig:pomdp}
\end{figure}

As presented in Figure~\ref{fig:pomdp}, our goal is a multi-class recognition policy $\pi$ that takes an image $\mathcal{I}$ and then outputs $\{\emph{classify}(1), \dots, \emph{classify}(K)\}$ and a list of multi-class detection results $\emph{detect}(I)$.

The policy repeatedly selects an action $a_i$ from a set of actions $\mathcal{A}$, executes it, potentially receives an observation $o_i$, and selects the next action.
The set of actions can include classifiers, detectors, or hybrid actions (detector followed by classification of its output).

A dynamic, or ``closed-loop,'' policy bases action selection on observations received from previous actions, exploiting the signal in inter-object and scene context for a maximally efficient path through the actions.
This is our goal, and what sets our formulation apart from multi-class systems that evaluate in a fixed order, such as simple cascades \cite{Viola2001} or decision trees.

Let $\mathcal{A}$ consist of $K$ detectors $L^\text{det}_i$.
We use one-vs-all deformable part-model classifiers on a HOG featurization of the image \cite{Felzenszwalb2010a}, with associated linear classification of the detections.

\comment{Let $\mathcal{A}$ consist of $K$ actions:
\begin{itemize}
  \item $K$ detectors $L^\text{det}_i$: one-vs-all deformable part-model classifiers on a HOG featurization of the image \cite{Felzenszwalb2010a}, with associated linear classification of the detections.
  \item $K$ classifiers $L^\text{gist}_i$: one-vs-all SVMs on the GIST feature \cite{Torralba2004}.
\end{itemize}}

\subsubsection{Time and Evaluation}
Each action $L$ has an expected cost $c(\cdot)$ of execution.
Depending on the setting, the cost can be defined in terms of algorithmic runtime analysis, an idealized property such as number of \emph{flops}, or simply the empirical runtime on specific hardware.
We take the empirical approach: every executed action advances $t$, the \emph{time into episode}, by its empirical runtime.

As shown in Figure~\ref{fig:evaluation}, the system is given two times: the setup time $T_s$ and deadline $T_d$.
From the setup time to the deadline, we want to obtain the best possible answer if stopped at any given time.
This corresponds to the general notion of \emph{Anytime} algorithms, and is motivated by desired flexibility in the system.

A single-number metric that corresponds to this objective is simply the ratio of the area captured under the curve to the total area between the start and deadline bounds.
\comment{Just like the metric of Average Precision itself was motivated by the inadequacy of any single Precision-Recall operating point to describe the performance of a robust system, our proposed metric is motivated by the inadequacy of any single Performance vs. Time operating point.}
We evaluate policies by this more robust metric and not simply by the final performance at deadline time for the same reason that Average Precision is used instead of a fixed Precision vs. Recall point.

\subsubsection{Sequential Execution}
An action $a_i$ that consists of running a classifier $L_i$ returns a real-valued observation $o_i \sim P(O_i)$.
The state records the fact that $a$ has been taken by adding it to the initially empty set $\mathcal{O}$.
We refer to the current set of observations as $\mathbf{o} = \{o_i | L_i \in \mathcal{O}\}$.

We define the belief state $b$ of the decision process by the the distribution over class presence variables $P(\mathbf{C}) = P(C_1, \dots, C_K)$, where we write $P(C_k)$ to mean $P(C_k=1)$.
Additionally, $b$ records the time into episode $t$, and the set of executed actions $\mathcal{O}$ with corresponding observations $\mathbf{o}$.

A recognition \emph{episode} takes an image $\mathcal{I}$ and proceeds from the initial belief state $b^0$ and action $a^0$ to the next pair $b^1$, and so on until $t$ exceeds $T_d$.
At that point, the policy is terminated and a new episode begins on a new image.

The policy's performance at time $t$ is determined by the detection and classification observations that have been observed at the last belief state $b^j$ before that point.
For classification of unobserved classes, we treat the corresponding values of $P(\mathbf{C})$ as the set of confidence scores $\{\emph{classify}(k)\}$.
Detection results of unobserved classes are an empty set.

Our notation is summarized in \autoref{tab:notation}.

\begin{table}[h!]
\centering
\caption{Summary of the notation.}
\label{tab:notation}
\begin{tabular}{|l|l|}
  \hline
  $\mathcal{I}$ & image \\
  $C_k$         & presence of class $k \in \{1,\dots,K\}$ \\ 
  $t$           & time into episode \\ 
  $T_s$, $T_d$  & start and deadline times \\ 
  $b^j$         & belief state at step $j$ \\ 
  $\pi$         & policy function, $b \mapsto a \in \mathcal{A}$ \\
  $\mathcal{A}$ & set of actions $a$\\ 
  \comment{$\mathcal{F}$ & set of featurization actions \\}
  \comment{$\mathcal{L}$ & set of classification actions\\}
  $o_i$         & a real-valued observation upon executing $a_i \in \mathcal{A}$\\
  $\mathcal{O}$ & set of executed actions\\
  $\mathbf{o}$  & set of observations $\{o_i | a_i \in \mathcal{O}\}$\\
  $c(a_i)$        & cost of executing $a_i$, in units of $t$\\
  \hline
\end{tabular}\end{table}

\section{Selecting actions} \label{sec:value}
\comment{Optimal performance results from becoming maximally certain of the correct values $C_i$ as quickly as possible (given the setup time).}
As our goal is to pick actions dynamically, we want to formulate a function $V(b,a)$  that assigns a value to a potential action, given the current state of the decision process.
We can then define the policy as simply the untaken action with the maximum value:
\begin{align}
\pi(b) = \argmax_{a_i \in \mathcal{A} \setminus \mathcal{O}} V(b,a_i)
\end{align}

For this policy to be \emph{closed-loop}, the observations $o_i$ generated by taking an action $a_i$ need to update $b$ and thus influence the selection of the next action.
Figure~\ref{fig:pomdp} visualizes such an execution process.

One simple heuristic value function that we can try is simply picking the action corresponding to the class presence variable $C_k$ with the highest probability.
Of course, we'd like to learn the value function from the data, but the above heuristic will be used as a baseline in the evaluation to follow.

Before we discuss how to set $V(b,a_i)$ such that our policy obtains best performance under our evaluation, we present our model for updating the belief state with observations.

\subsection{The model of the belief state}
\begin{figure}[h!]
\centering
\includegraphics[width=0.56\linewidth]{../figures/inf_model_mrf_1.pdf}
\caption{
The MRF inference model used in our system.
We show a point in the middle of the decision process for 3 classes; one action has already been taken.
\comment{Some classifiers have already been observed.}
}
\label{fig:model}
\end{figure}

The quantities that may be useful to us for selecting which actions to deploy are the probabilities and entropies of the class presence variables $C_k$.
These allow us to look for the most probable classes given the observations.
When the policy starts, the model should present the prior distributions $P(C_k)$; as observations are accrued, the model should present the updated conditionals $P(C_k|\mathbf{o})$.

We employ a fully-connected Markov Random Field (MRF), as shown in Figure~\ref{fig:model}.
The $L_i$ variables are discretized from real-valued responses of a classifier on the detections output by the deformable part-model detector we employ (see Section~\ref{sec:tech}).

The classifier is a linear kernel SVM on the top two max detection scores in the list of detections.
The responses are discretized per variable based on the distribution of the scores on the training dataset.

The MRF is implemented with an open-source graphical model package \cite{Jaimovich2010}.
The parameters of the model are trained on fully-observed data.
Exact inference is generally intractable in this model.
Instead, we use Loopy Belief Propagation, which does not provide general convergence guarantees but has been shown to work well empirically on similar tasks \cite{Desai2009}.

\subsection{Learning the Value Function}

Remember that our heuristic value function consisted of picking the action corresponding to the class presence variable with the highest probability $P(C_k)$.
We can formulate such a policy as a scalar product:
\begin{align}
\pi(b) = \argmax_{a_i \in \mathcal{A} \setminus \mathcal{O}} \theta^\top \phi(b,a_i)
\end{align}
where $\phi(b,a_i)$ is a feature vector representation of the belief state.
This approach is known as function approximation in reinforcement learning \cite{Sutton1998}.

Let us take take the feature representation for an action $a_i$ to be $[P(C_k), \, 1]$, where $C_k$ corresponds to the action.
This corresponds to our heuristic value function, with a bias variable.
The feature representation $\phi(b,a_i)$ then is a vector of size $F|\mathcal{A}|$, with $F=2$ for this feature, where all values are $0$ except those corresponding to $a_i$.

This representation allows us to use a single vector of weights $\theta$ as a compact representation of our policy.

At each time step, our system is evaluated by properties of the state $b$ (the list of detections and the classification outputs).
The final evaluation metric is a function of the history of execution $h^0=b^0,b^1,\dots,b^J$, with $J$ being the last step of the process with $t \le T_d$.

Ideally, the value function for a point in the decision process $b^j$ should give the expected value of the final evaluation metric, over all possible histories starting at point $j$:
\begin{align}
V(b^j,a_i) = \mathbb{E}_{h^j \sim P(h^j|b,a_i)}[R(h^j)]
\end{align}

The reward function $R(h^j)$ assigns a real-valued score to a history.
We have considerable flexibility in defining $R$; it does not necessarily have to be directly tied to the final evaluation.

One feature we want the reward function to have is additivity.
Let's say that given deadline $T_d$ and some image, the policy had time to take $J$ actions.
The total reward of a policy $\pi$ starting at state $b^j$ is then defined as the sum of rewards
\begin{equation}
R(h^j) = \sum_{j'=j}^J R(b_j',a^{j'})
\end{equation}

We formulate two reward definitions: one strives to match the final AP evaluation of the detections, and one is motivated by lowering uncertainty of $P(\mathbf{C})$.

\subsubsection{Reward: area under the AP vs. Time curve}
\begin{figure}[htb]
  \centering
  \includegraphics[width=0.56\linewidth]{../figures/apvst_expl.pdf}
  \caption{A per-action greedy value function that corresponds to the maximization of our objective function is the area of the horizontal slice under the curve due to the action. The figure shows this analysis for the action highlighted in orange.}
  \label{fig:rewards}
\end{figure}

The final evaluation of a policy consists of the area under the performance vs. time curve (normalized by the total area).
Accordingly, we formulate per-action rewards such that their addition results in this quantity.

Specifically, as shown in Figure~\ref{fig:rewards}, we define the reward of an action as
\begin{equation}\label{eq:advanced}
R(b^j,a_i) = \frac{\Delta \text{ap}_i (t_T^j-\frac{1}{2}\Delta t_i)}{(1-\text{ap}^j)t_T^j}
\end{equation}
where $t_T^j$ and $\text{ap}^j$ are the time left until deadline and the AP at state $b^j$, and $\Delta t_i$ and $\Delta \text{ap}_i$ are the time taken and AP change produced by the action $a_i$.

The reward is $1$ if the policy takes an action that obtains the maximum possible area under the curve at that point, and $0$ if no area under the curve is captured.

\subsubsection{Reward: decrease in mean entropy}
We also consider a similar additive reward function based on the mean entropy of the variables $P(C_k)$: $\frac{1}{K}\sum_{k=1}^K H(C_k)$, where $H(C_k) = - \sum_{c_k \in {0,1}} P(c_k) \log P(c_k)$.
We follow the same setup as above, but strive to maximize the area \emph{above} the curve of mean entropy vs. time.

The goal of a policy that maximizes these rewards is to take actions that reduce uncertainty of the most uncertain variables most quickly.

\subsection{Learning the weights}

The feature representation of the belief state for a given action $a_i$ that corresponds to class $k$ is defined to be
\begin{align}
\phi_k(b) = [P(C_k), \, H(C_k), \, 1]
\end{align}

Our procedure for learning the weights is standard generalized policy iteration \cite{Sutton1998}.
We first initialize the weights $\theta$ to the heuristic value function of simply picking the maximum $P(C_k)$: $\theta_i = [1, \,0, \,0]$.

With this policy, we run $N$ recognition episodes.
From the state-action samples gathered in running the episodes, we formulate a matrix $\Phi$ from the featurizations $\phi(b^j,a_i)$, and a vector $y$ consisting of the individual rewards $R(b^j,a_i)$.

The rewards are computed as the sum of discounted rewards to the end of the episode:
\begin{align}
R(b^j,a_i) = \sum_{i=0}^{J-j} \lambda^i R(b_{j+i},a^{j+i})
\end{align}
Note that with $\lambda=0$, the reward is determined entirely by the actual action taken; with $\lambda=1$, the reward is the sum of all rewards until the end of the episode.

We then solve the system $\Phi \theta = y$ for $\theta$ with Lasso regression.
With the updated weights $\theta$, we run $N$ more episodes and repeat the procedure until convergence.

The parameters of this training procedure are the $\alpha$ weight on the regularization term in the regression and the $\lambda$ discount weight.
We cross-validate for both values.

\section{Time-sensitive evaluation} \label{sec:evaluation}
Average Precision (AP) has become a standard evaluation for detector performance on challenging datasets.
AP is the area under the Precision vs. Recall curve, which is obtained by varying the threshold on the confidence of the detector.

Just as we care about the performance of our system at different thresholds of detection, we should also care about performance as the system is given different amount of time to output detections.
Accordingly, we plot the P-R \emph{surface} instead of the P-R curve, with time as one of the axes.
The surface can be integrated along the Recall dimension to yield an AP vs. time curve.

The goal of our scheme is to schedule computation optimally, so that the largest part of the detection performance is recovered early on.
In the limit of infinite time, we still want optimal performance---the performance of our system should not degrade as it is given more time.
Accordingly, our goal is \emph{Anytime} performance starting at some fixed deadline.

As our task is fundamentally in \emph{multi-class} object detection, we rely on a slightly different evaluation than is commonly used: instead of pooling detections across images in the dataset but not classes, we pool detections across classes, but evaluate per-image (and report the average).
This has been done before~\cite{Desai2009}.

\pagebreak
\section{Conclusion}
\vspace{-.1in}
We presented a method for learning ``closed-loop'' policies for multi-class object recognition, given existing object detectors and classifiers and a metric to optimize.
The method learns the optimal policy using reinforcement learning, by observing execution traces in training.
If detection on an image is cut off after only half the detectors have been run, our method does $66\%$ better than a random ordering, and $14\%$ better than an intelligent baseline.
In particular, our method learns to take action with no intermediate reward in order to improve the overall performance of the system.

As always with reinforcement learning problems, defining the reward function requires some manual work.
Here, we derive it for the novel detection AP vs. Time evaluation that we suggest is useful for evaluating efficiency in recognition.
Although computation devoted to scheduling actions is less significant than the computation due to running the actions, the next research direction is to explicitly consider this decision-making cost; the same goes for feature computation costs.
Additionally, it is interesting to consider actions defined not just by object category but also by spatial region.
The code for our method is available\footnote{\url{http://sergeykarayev.com/work/timely/}}.

\subsubsection*{Acknowledgments}
This research was made with Government support under and awarded by DoD, Air Force Office of Scientific Research, National Defense Science and Engineering Graduate (NDSEG) Fellowship, 32 CFR 168a.

\renewcommand\bibsection{\subsubsection*{\refname}}
\bibliographystyle{unsrt}
\small{
  \bibliography{../sergeyk_Timely}
}

\end{document}
