\documentclass[11pt]{article}
\usepackage[utf8]{inputenc}
\pagestyle{headings}
\usepackage[top=1in, bottom=1in]{geometry}
\usepackage{amsmath,amsfonts,amsthm,amssymb}
\usepackage[pdftex]{graphicx}
\usepackage{url}
\usepackage{subfig}
\usepackage{color}
\usepackage[pagebackref=true,breaklinks=true,letterpaper=true,colorlinks,bookmarks=false,citecolor=red,linkcolor=blue]{hyperref}

%% Command definitions
\def\subsectionautorefname{section}
\newcommand{\note}[1]{\textcolor{red}{\textbf{#1}}}
\definecolor{light-gray}{gray}{0.3}
\newcommand{\aside}[1]{\textcolor{light-gray}{\emph{#1}}}
\newcommand{\comment}[1]{}

\title{Attentional Object Detection: A Proposal}
\author{Sergey Karayev}
\date{updated: 22 April 2011}

\begin{document}
\maketitle

\begin{abstract}
This document tracks the development of ideas for efficient object detection using the idea of attention: a sequential process of looking for something somewhere.
We follow the Deformable Part Model approach to object detection and extend it in several ways, mostly focusing on the role of context, toward the goal of Anytime detection performance.
I expect a submission to NIPS'11 to be whittled down from the text and results here.
\end{abstract}

\tableofcontents
\newpage

%!TEX root=paper/paper.tex
\chapter{Introduction}\label{sec:introduction}

\section{Motivation}

\PM{Perception}
It is well known that human perception is both Anytime, meaning that a scene can be described after even a short presentation, and progressive, meaning that the quality of description increases with more time.
The progressive time course of visual perception has been confirmed by multiple studies \parencite{Vanrullen-1996,Fei-Fei-Vision-2007}, with some studies providing evidence that enhancement occurs in an ontologically meaningful way.
For example, people tend to recognize something as an animal before recognizing it as a dog \parencite{Mace-PloS-2009}.
The underlying mechanisms of this behavior are not well explored, with only a few attempts made to explain the temporal dynamics --- for instance, a promising work by \cite{Hegde-Neuro-2008} has employed the framework of sequential decision processes.

\PM{Computer applications}
Meanwhile, automated visual recognition has achieved levels of performance that allow useful real-world implementation.
We focus on two problem formulations: \emph{image classification}, in which some property of the image -- such as scene type, visual style, or even object presence -- is predicted, and \emph{object detection}, in which the location and category (or identity) of all objects in a scene is predicted.
Solutions to the two problems are often linked, as classification can be a ``subroutine'' in a detection method.
State-of-the-art methods for classification and detection tend to be computationally expensive, insensitive to Anytime demands, and not progressively enhanced.

\PM{Application}
As real-world deployment of recognition methods grows, managing resource cost (power or compute time) becomes increasingly important.
For tasks such as personal robotics, it is crucial to be able to deploy varying levels of processing to different stimuli, depending on computational demands on the robot.
A hypothetical system for vision-based advertising, in which paying customers engage with the system to have their products detected in images on the internet, presents another example.
The system has different values (in terms of cost per click) and accuracies for different classes of objects, and the backlog of unprocessed images fluctuates based on demand and available server time.
A recognition strategy to maximize profit in such an environment should exploit all signals available to it, and the quality of detections should be Anytime, depending on the length of the queue (for example, lowering recall with increased queue pressure).

\PM{Visual Features \& Classification}
For most state-of-the-art classification methods, a range of features are extracted from an image instance and used to train a classifier.
Since the feature vectors are usually high-dimensional, linear classification methods are used most often.
Features are extracted at different costs, and contribute differently to decreasing classification error.
Although it can generally be said that ``the more features, the better,'' high accuracy can of course be achieved with only a small subset of features for some instances.
Additionally, different instances benefit from different subsets of features.
For example, simple binary features are sufficient to quickly detect faces \parencite{Viola-IJCV-2004} but not more varied visual objects, while the features most useful for separating landscapes from indoor scenes \parencite{Xiao-CVPR-2010} are different from those most useful for recognizing fine distinctions between bird species \parencite{Farrell-ICCV-2011}.
\autoref{fig:features} presents several common visual features.

\PM{Detection}
Detection methods tend to employ the same visual features and classifiers but apply them to many image sub-regions.
Approaches can broadly be grouped into (A) \emph{per-class, all-region}, (B) \emph{all-class, all-region}, and (C) \emph{all-class, proposed-region} methods.
State-of-the-art \emph{all-class, proposed-region} methods such as \cite{Girshick-CVPR-2014} and \emph{per-class, all-region} methods such as \cite{Felzenszwalb2010a} are considerably slow, performing an expensive computation on (respectively) a thousand to a million image windows.
To maximize early performance gains of these methods, scene and inter-object contextual cues can be exploited in two ways.
First, regions can be processed in an intelligent order, with most likely locations selected first.
Second, if detectors are applied per class, then they can be sequenced so as to maximize the chance of finding objects actually present in the image.
And even the most recent \emph{all-class, all-region}, Convolutional Neural Net (CNN)-based detection methods such as \cite{He-ECCV-2014}, which take advantage of high-performance convolutional primitives for region processing and detect for all classes simultaneously, can be sped up using our idea of cascaded classification.

\section{Our Contributions}

\PM{Costliness}
Computing all features, running all detectors, or processing all regions for all images is infeasible in a deployment sensitive to Anytime needs, as each feature brings a significant computational burden.
Yet the conventional approach to evaluating visual recognition does not consider efficiency, and evaluates performance independently across classes.
We address the problem of selecting and combining a subset of features under an Anytime cost budget (specified in terms of wall time or total power expended or another metric) and propose a new \emph{costliness} measure of performance vs. cost.

\PM{Learning a Policy}
To exploit the fact that different instances benefit from different subsets of features, our approach to feature selection is a sequential policy.
To learn the policy parameters, we formulate the problem as a Markov Decision Process (MDP) and use reinforcement learning methods.
The method does not make many assumptions about the underlying actions, which can be existing object detectors and feature-specific classifiers.
With different settings of parameters, we can learn policies ranging from \textbf{Static, Myopic}---greedy selection not relying on any observed feature values, to \textbf{Dynamic, Non-myopic}---relying on observed values and considering future actions.
The foundational machinery is laid out in \autoref{sec:det_method}.

\PM{Per-class Detection}
For \emph{per-class} detection, the actions are time-consuming detectors applied to the whole image, as well as a quick scene classifier.
We run scene context and object class detectors over the whole image sequentially, using the results of detection obtained so far to select the next actions.
Since the actions are time-consuming, we use a powerful inference mechanism to select the best next action.
In \autoref{sec:det_evaluation}, we evaluate on the PASCAL VOC dataset and obtain better performance than all baselines when there is less time available than is needed to exhaustively run all detectors.
This work was originally presented in \cite{Karayev-NIPS-2012} and all work is open source\footnote{Available at \url{https://github.com/sergeyk/timely_object_recognition}}.

\PM{Image Classification}
Classification actions are much faster than detectors, and the action-selection method accordingly needs to be fast.
Because different features can be selected for different instances, and because our system may be called upon to give an answer at any point during its execution, the feature combination method needs to be robust to a large number of different observed-feature subsets.
In \autoref{sec:clf_chapter}, we consider several value-imputation methods and present a method for learning several classifiers for different clusters of observed-feature subsets.
We first demonstrate on synthetic data that our algorithm learns to pick features most useful for the specific test instance.
We demonstrate the advantage of non-myopic over greedy, and of dynamic over static on this and the Scene-15 visual classification dataset.
Then we show results on a subset of the hierarchical ImageNet dataset, where we additionally learn to provide the most specific answers for any desired cost budget and accuracy level.
This work was originally presented in \cite{Karayev-CVPR-2014} and all work is open source\footnote{Available at \url{https://github.com/sergeyk/anytime_recognition}}.

\PM{Cascade CNN}
We additionally investigate a novel approach for speeding up a state-of-the-art CNN-based detection method, and propose a general technique for accelerating CNNs applied to class imbalanced data.
We employ the classic idea of the cascade by inserting a \emph{reject} option between expensive convolutional layers.
When a CNN processes batches of images, which is standard for many applications, the reject layers allows ``thinning'' of the batch as it progresses through the network, thus saving processing time.
This method is applicable to both \emph{all-class, proposed-region} methods such as \cite{Girshick-CVPR-2014} and \emph{all-class, all-region} methods such as \cite{He-ECCV-2014}.
We demonstrate results --- along with a variety of strong baselines -- on the former method, and show that the Cascade CNN method obtains a nearly 10x speed-up with only marginal drop in accuracy.
All work is reported in \autoref{sec:ccnn_chapter}.

\PM{Recognizing Style}
Lastly, in \autoref{sec:style_chapter} we present two novel datasets and first results for an underexplored research problem in computer vision -- recognizing visual style.
In preparation for an Anytime approach, we evaluate several different features (including CNNs) for the task, and explore content-style correlations in our datasets.
Our large-scale learning gives state-of-the-art results on an existing dataset of image quality and photographic style, and provides a strong baseline on our contributed datasets of 80K photos and 85K paintings labeled with their style and genre.
In a demonstration of cross-dataset understanding of style, we show how results of a search by content can be filtered by style.
This work was originally presented in \cite{Karayev-BMVC-2014}, and all code is open source\footnote{Available at \url{https://github.com/sergeyk/vislab}}.

\PM{Future Directions}
This thesis provides an effective foundation for Anytime visual recognition, and points the way to interesting further developments.
Our MDP-based formulation of learning a feature-selection policy is empirically effective, but heuristic in nature.
The recently developed framework of adaptive submodularity \parencite{Golovin-and-Krause-2010-JAIR} could provide theoretical near-optimality results for some policies, but developing an appropriate objective for our task is not straightforward.
We showed our Cascade CNN model to be effective for a region-based detection task -- but the model was not trained end-to-end with the threshold layers.
An even more interesting future development would add an Anytime loss layer that combines classification output from multiple levels of the network in a cost-sensitive way.
We expand on these ideas in \autoref{sec:conclusion}.

%!TEX root=paper/thesis.tex

\section{Time-sensitive evaluation} \label{sec:evaluation}
Average Precision (AP) has become a standard evaluation for detector performance on challenging datasets.
AP is the area under the Precision vs. Recall curve, which is obtained by varying the threshold on the confidence of the detector.

Just as we care about the performance of our system at different thresholds of detection, we should also care about performance as the system is given different amount of time to output detections.
Accordingly, we plot the P-R \emph{surface} instead of the P-R curve, with time as one of the axes.
The surface can be integrated along the Recall dimension to yield an AP vs. time curve.

The goal of our scheme is to schedule computation optimally, so that the largest part of the detection performance is recovered early on.
In the limit of infinite time, we still want optimal performance---the performance of our system should not degrade as it is given more time.
Accordingly, our goal is \emph{Anytime} performance starting at some fixed deadline.

As our task is fundamentally in \emph{multi-class} object detection, we rely on a slightly different evaluation than is commonly used: instead of pooling detections across images in the dataset but not classes, we pool detections across classes, but evaluate per-image (and report the average).
This has been done before~\cite{Desai2009}.
\section{Problem Formulation}
Our goal is to output the best detections at some deadline $F$, and to continue having the best detections at every time step after that.
Specifically, we seek to maximize the average multi-class AP of a detector for all classes in all images of a test dataset.

Given enough time, our program will look everywhere for everything, but given a short deadline, it should first look in the most promising places for the most likely objects.
To take advantage of semantic and spatial context, where the program looks at time $t$ should depend on the results gathered by $t-1$.

In each of the $I$ images we consider, we look for $K$ classes of objects, and have detectors $c_k$.

We represent an image as a collection of overlapping windows at different scales: a multi-scale pyramid of $N$ locations $\l_i = (x,y,s)$.
We assume that each location $\ell$ can be the center of at most one bounding box, containing an object belonging to one of $K$ classes.

We can consider three variants of object detection as a sequential decision problem:
\begin{enumerate}
  \item Looking for everything at a location: Selecting a location $i$, and running a black box program $P$ there that evaluates all detectors as it sees fit.
  \item Looking for something at a location: Selecting a location $i$ and a detector $c_k$. Picking the sequence of partial classifiers for $d$ is then a black box subproblem.
\item Looking for partial evidence of something at a location: Selecting a location $i$ and a partial classifier $c_{kp}$.
\end{enumerate}
We will be working at the level of the second problem---we assume that we have $K$ powerful classifiers, and want to pick the shortest path through (location, class) pairs to all correct detections.

\subsection{POMDP}
We face a Partially Observed Markov Decision Process (POMDP), which we define as the tuple $(S,A,O,b_o,T,\Omega,R,\gamma,F)$, where:
\begin{itemize}
  \item $S$ is a set of discrete states, $A$ is a set of discrete actions, and $O$ is a set of continuous observations.
  \item $b_o$ is the initial belief state.
  \item $T(s,a,s') = P(s_{t+1} = s | a_t = a, s_t = s)$ is the distribution describing the probability of transitioning from state $s$ to state $s'$ upon taking action $a$.
  \item $\Omega(o,s,a) = P(o_{t+1}=o | a_t = a, s_{t+1} = s)$ is the distribution describing the probability of observing $o$ from state $s$ after taking action $a$.
  \item $R(s,a)$ is the reward signal received when executing action $a$ in state $s$.
  \item $F$ is the deadline to termination.
\end{itemize}
Our goal is to learn the optimal policy $\pi$, which is a mapping from belief states to actions.

\subsection{State representation}
For all approaches described later, the state $s \in S$ is fully defined by the time $T$, and the class of the bounding box centered at each location.
We augment the $K$ classes with a ``background'' class.
Note that each object in the image is assigned to exactly one location.
The number of states is exponential in $N$, which is on the order of $10000$ in our applications.\comment{$T (K+1)^N$ states.}
We are dealing with a static image and our actions do not change the underlying state, so $T(s,a,s') = \delta_{s, s'}$ where $\delta$ is the Kronecker delta function.

The belief state is supposed to be a distribution over ``physical'' states.
We model the belief state as a Conditional Random Field (CRF) over the $N$ locations in the image pyramid, where a node $y_{i} \in Y$ is an integer $0 \dots K$, according to the class whose bounding box we believe is centered at location $i$ (background is $0$).

For each node $y_i$, there are $\sum_{k=1}^K P_k$ nodes $z_{ikp}$, which represent a classifier's confidence in the corresponding $y_i$ node \note{(in some way)}.
Two nodes $y_i$ and $z_{ikp}$ are connected with weight $w_{kp}$.
This weight in a sense represents our confidence in the classifier $c_kp$.
We also leave open the possibility of extra nodes $z_{ikp'}$ to model contextual cues or \emph{a priori} beliefs.

The connections between two nodes $y_i$ and $y_j$ are defined by the spatial context feature $d_{ij}$ (represented in~\autoref{fig:dij}) and the weights $w_{y_i,y_j}$, which encode valid geometric configurations of object classes $y_i$ and $y_j$.
\begin{figure}[h!]
  \caption{Figure from~\cite{Desai2009}.}
  \centering
    \includegraphics[width=0.7\textwidth]{../figures/dij.pdf}
  \label{fig:dij}
\end{figure}

This way of modeling the underlying state of the image is similar to the one used in~\cite{Desai2009}.
In fact, we rely on their method of greedy forward search to do inference in the CRF and learn the parameters; inference of the physical state from our belief state is described in~\autoref{sec:rewards}.

\subsection{Actions and Observations} \label{sec:actions}
Actions are defined by a choice of location $i$ and partial detector $c_{kp}$, making for $N \sum_{k=1}^K P_k$ possibilities.

Upon executing an action, the observation $o$ we get back is the decision of the classifier (or its score, if provided).
The observation updates the corresponding node $z_{ikp}$.
We can propagate this update through the CRF with an iteration of belief propagation, or alternatively, we can fully ``resolve'' our CRF by running a greedy forward search (details in~\autoref{sec:rewards}).
We can do this at every action, every fixed number of actions, or upon taking a special action.

We may also want to think of executing actions in batches, such as ``evaluate $c_{kp}$ at all locations,'' or ``evaluate all $c$ at a location $i$.''
We can represent such sets of actions as a special action.

\subsection{Deadline} \label{sec:deadlines}
We express $F$ in units of expected time per action, such that taking action $a$ with time cost $\tau$ advances $T$ by $\tau$.
We determine $\tau$ by averaging the wall clock time of executing $a$; this is done for all $a$ prior to running the POMDP learner.

Toward our goal is Anytime performance starting at some fixed deadline $F$ (see~\autoref{sec:evaluation}), our training procedure enforces the deadline stochastically, terminating the agent at any random $T \geq F$.

\subsection{Outputting Detections and Rewards} \label{sec:rewards}
$Y$ in our belief state represents the detections, and $Z$ provides confidences for $Y$.

Recall that we cannot have detections of two different classes at the same location, and that the PASCAL evaluation will penalize all but one detections of the same class at overlapping locations.
For this reason, most detection systems that ``assign content to locations'' run a post-processing non-maximum suppression step~\cite{Felzenszwalb2010a}.
In a multi-class setting, it is better to resolve such ambiguities with a principled system to model both inter- and intra-class, and short- and long-range interactions.

Our belief state is modeled after such a system, and we simply run a greedy forward search to maximize the energy of our CRF to pick our final detections~\cite{Desai2009}.
The energy of the CRF is given as
\begin{equation}
  S(Y,Z) = \sum_{i,j} w_{y_i,y_j}^T d_{ij} + \sum_{i,kp} w_{kp} z_{ikp}
\end{equation}
In brief, the greedy search starts out with an empty set of detections, and proceeds to pick those $y_i$ that maximize $\Delta S(Y,Z)$.
It was shown to be optimal for nearly all images ($\approx98\%$) in the PASCAL set (and compared against Loopy BP and Tree-ReWeighted BP)~\cite{Desai2009}.

The theoretical explanation for this comes from the work on \emph{submodular functions}.
The rough idea is that to maximize functions where adding an element to a large set has less effect than adding an element to a small set, greedy algorithms often have theoretical guarantees of near-optimal performance.
The more pairwise connections are non-positive in the CRF, the more submodular it is.
\note{Mention percentage of our learned connections that are non-positive to justify this.}

After every action, $R(s,a)$ is assigned according to the AP of these detections.
The program is terminated at time $F$.

\subsection{Some Example Policies}
To get a feel for how different detection strategies fit into our framework, we go through some examples, in order of increasing complexity.

\paragraph{Sliding window}
Here, we take actions that run whole detectors and not their individual classifiers.
We featurize the belief state with just two things: the location $i$ and class $k$ of the last action.
The policy simply increments these in order, cycling through either locations or classes.

\paragraph{Sliding window, cascaded}
Now we pick the actual classifier, not just the detector, so let's store $i$ and $c_{kp}$ of the last action.
The policy can use the thresholds that the Cascaded DPM detector learned: if observation $o$ is past threshold for $c_{kp}$, then the next action will be $c_{k{p+1}}$ at the same location; otherwise, we move on to another location or class.

\paragraph{Saliency-driven}
Once again we take actions that run whole detectors.
Our policy first computes a simple saliency map for the image (for example, as in~\cite{Alexe2010}).
The belief state is featurized by the location of the current unexamined (picked less than $K$ times) max in the saliency map and the class $k$ of the last action.
The policy is to simply sample locations in order of saliency, running all detectors at a location before proceeding to the next.

\paragraph{Class prior-driven}
On the training set, we compute $K$ canonical object likelihood maps.
We featurize the belief state with the current unexamined (picked less than $1$ time) max of each likelihood map.
Our policy is to run the detector for the class $k$ at the location $i$ with the largest likelihood among these maps.

\paragraph{Root model score-driven, cascaded}
The first stage of the cascaded DPM detector is a PCA-reduced root filter.
Our policy has two stages.
First, it follows a modified \textbf{sliding window} policy to run the fast root filter everywhere in the image, for all classes.
Then, it follows a modification of the \textbf{Class prior-driven} policy, with the class priors given by the root filter scores, and classifier actions picked as in \textbf{Sliding window, cascaded}.

\paragraph{Coarse-to-fine, inspired by~\cite{Pedersoli2011}}
As above, except in between the stages, we resolve our belief state CRF (which basically does NMS in addition to other things).
This is similar to what they do in~\cite{Pedersoli2011}: run NMS after root filter scoring, then run higher-resolution parts and resolve them again, although we don't do the latter.

\paragraph{Coarse-to-fine, with smart featurization}
As explained in~\autoref{sec:discretization_and_featurization}, we can use low-resolution templates and the lower-scale part of the image pyramid to get estimates on the detections in the higher-scale part of the pyramid.

Our policy first computes only the lower scales of the feature pyramid, and otherwise follows the \textbf{Root model score-driven, cascaded} policy with a low-res root filter.
\aside{
\paragraph{Multi-scale models}As explained in~\autoref{sec:discretization_and_featurization}, it has been shown that one should not run high-resolution part-based models to look for small objects in the image pyramid~\cite{Park2010}.
Their approach learns when to use a low-resolution model and when to use a high-res model, and trains the combined model jointly.
Deciding when to use the low-res model could be part of a policy.
}
\section{Location Discretization and Feature Computation} \label{sec:discretization_and_featurization}

\paragraph{Location Discretization}
The fewer possible discrete locations $l_i = (x,y,s)$, with $i = 1 \dots N$, we consider, the smaller our state space and therefore the better our reinforcement learning solution for a given amount of computation.
Featurization also takes less time as tolerable coarseness increases.
At the same time, we lose power to match ground truth detections.

There are three factors in the discretization process: the stride of the center point of the window $(x,y)$, and the number of octaves and scales per octave of $s$.

\note{todo: include a figure plotting oracle performance vs. decreasing coarseness of the image pyramid discretization.}

\paragraph{Coarse-to-Fine Evaluation}
The standard scale-invariant approach to detection with a template is to evaluate a template of fixed size over different scales of the image pyramid.
A recent report makes the observation that this approach using high-resolution, part-based models performs worse than scale-variant detection that degrades to using low-resolution, rigid models for detection at small scales of the image~\cite{Park2010}.
They demonstrate state-of-the-art results on a pedestrian detection benchmark.
The authors also note that context is most helpful for small-object detections.

\begin{figure}[h!]
  \caption{Illustration of the idea of using low-res models to approximate the output of high-res models, allowing us to only compute the image pyramid at small scales.}
  \centering
    \includegraphics[width=0.7\textwidth]{../figures/multiscale.pdf}
  \label{fig:multiscale}
\end{figure}
We note also that a low-resolution model used at scale $s/2$ provides an approximation to the output of a high-resolution model at scale $s$, as illustrated in~\autoref{fig:multiscale}.
This may allow us to stagger feature computation, first computing just the small-scale part of the pyramid and running low-res models, and then computing the the large-scale part of the pyramid as time allows.

\paragraph{Feature Computation}
\note{todo: how can we include feature computation as part of the action space?}

\note{should look at Piotr Dollar's work on efficient HOG computation~\cite{Dollar2010}.}
\section{Loose Ideas}

\subsection{Local context feature}
We could try augmenting the DPM model with a local context window, for example by learning another HOG template on a window that is slightly larger than the object window.
Has this been done, and why not?
It's also what Allie is thinking of doing for shadow (shadow context around object windows).

\subsection{Fast multi-class approximation of the right class} \label{sec:multiclass_approx}
If we consider a window and have $N$ detectors, how can we efficiently figure out which of the $N$ detectors have the greatest likelihood of giving a high score to this window?

One idea is to use classifier dimensionality reduction.
Let's assume we are using SVM classifiers and classifying vectors of the same dimension.
Do we really need to run all of them in their full dimensions, to have a good guess as to which are going to score high?
Could we not use dimensionality reduction on all the learned classifiers to come up with a very low-dimensional evaluation that would approximate the full-dimensional score?
Then we could only run those classifiers that are past threshold on this low-dimensional approximation.

Another idea is to forget about the specific classifier we are using and just try to partition the feature space into class clusters, roughly.
We can use K-means, for example.
The cluster(s) that a specific window then falls into determine the more complex classifiers that we will evaluate.

\aside{
A similar idea is explored in the paper \cite{Isukapalli2006}.
They use boosting to train a cascade of classifiers to classify all object classes.
Then they learn a policy to partition the output of the detectors into class-specific clusters.
The policy is a decision tree.
The leafs of the decision trees specify the expected class of that window, and they use that to run a class-specific classifier cascade.
}
\section{Results}

We want to show two things: 1) given the same amount of time as our baseline systems, we perform at least as well as them; 2) in the AP vs. time evaluation, we do better than all baselines.

\begin{description}
  \item[Cascaded DPM Baseline~\cite{Felzenszwalb2010b}] We should do as well as it does in the same time ($\approx$0.5 sec), and better than it for shorter times than that.
  \item[Coarse-to-fine Baseline~\cite{Pedersoli2011}] Same thing.
\end{description}

We should think of ways to make the comparison fair.

Should we run the multi-class NMS~\cite{Desai2009} over the detections of the baselines?
Since our approach relies on it, it seems that we should; on the other hand, it's part of the special sauce of our approach.

To make sliding window detectors run faster, we can modify the stride or scale quantization of their sliding window.
So, if we want to output detections in 0.1 seconds, we can modify the Cascaded DPM code, for example, to cover the image in that long.

\aside{
We should also evaluate on some kind of video object detection, to showcase the attentional priors.
Could be an easy CVPR'12 submission.
}

\bibliographystyle{ieeetr}
\small
\bibliography{sergeyk-bibtex}
\end{document}
